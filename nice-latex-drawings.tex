\documentclass[10pt,fleqn]{article}

\usepackage{algorithm}
\usepackage{algorithmicx}
\usepackage[noend]{algpseudocode}
\usepackage{amsfonts}
\usepackage{amsmath}
\usepackage{amsthm}
\usepackage[ngerman]{babel}
\usepackage[left=3cm, right=3cm]{geometry}
\usepackage{hyperref}
\usepackage{listings}
\usepackage{mathtools}
\usepackage{tikz}
\usetikzlibrary{arrows, calc, positioning, snakes}

\def\distance{\baselineskip}
\newenvironment{heading}{\begin{center}\bfseries}{\end{center}}

\begin{document}
    \begin{minipage}{\linewidth}
        \centering
        \begin{heading}
            Waiting Queue
        \end{heading}

        \begin{tikzpicture}
            % Axis
            \draw (-0.7, 0.3) node{Zeit} (-0.7, 0.3);
            \draw (0,  0.3)  node{0}  (0,  0.3);
            \draw (2,  0.3)  node{10} (2,  0.3);
            \draw (4,  0.3)  node{20} (4,  0.3);
            \draw (6,  0.3)  node{30} (6,  0.3);
            \draw (8,  0.3)  node{40} (8,  0.3);
            \draw (10, 0.3) node{50} (10, 0.3);
            \draw (12, 0.3) node{60} (12, 0.3);
            \draw (14, 0.3) node{70} (14, 0.3);
            \draw (0, 0) -- (14.8, 0);
            \draw[snake=ticks, segment length=2cm] (0, 0) -- (14.8, 0);
            % K1
            \draw (-0.5, -0.55) node[black]{K1} (0.5, -0.55);
            \draw[thick, fill=gray!25] (0,   -0.3) rectangle node{1}  (1.2,  -0.8);
            \draw[thick, fill=gray!25] (1.2, -0.3) rectangle node{4}  (3.2,  -0.8);
            \draw[thick, fill=gray!25] (3.2, -0.3) rectangle node{6}  (5.8,  -0.8);
            \draw[thick, fill=gray!25] (5.8, -0.3) rectangle node{9}  (9.8,  -0.8);
            \draw[thick, fill=gray!25] (9.8, -0.3) rectangle node{12} (10.6, -0.8);
            \draw[thick, fill=gray!25] (10.6,-0.3) rectangle node{14} (14.8, -0.8);
            % K2
            \draw (-0.5, -1.35) node[black]{K2} (0.5, -1.35);
            \draw[thick, fill=gray!25] (0,   -1.1) rectangle node{2}  (3.0,  -1.6);
            \draw[thick, fill=gray!25] (3.0, -1.1) rectangle node{5}  (3.6,  -1.6);
            \draw[thick, fill=gray!25] (3.6, -1.1) rectangle node{7}  (6.6,  -1.6);
            \draw[thick, fill=gray!25] (6.6, -1.1) rectangle node{11} (10.4, -1.6);
            \draw[thick, fill=gray!25] (10.4,-1.1) rectangle node{13} (11.6, -1.6);
            % K3
            \draw (-0.5, -2.15) node[black]{K3} (0.5, -2.15);
            \draw[thick, fill=gray!25] (0,  -1.9) rectangle node{3}  (4.6,  -2.4);
            \draw[thick, fill=gray!25] (4.6,-1.9) rectangle node{8}  (6.2,  -2.4);
            \draw[thick, fill=gray!25] (6.2,-1.9) rectangle node{10} (14.2, -2.4);
        \end{tikzpicture}
    \end{minipage}

    \begin{minipage}{\linewidth}
        \centering
        \vspace{\distance}
        \begin{heading}
            On the Post Correspondence Problem
        \end{heading}

        \begin{tikzpicture}[
            semithick
        ]
            \draw (0, 0) rectangle (1, -0.5);
            \node at (0.5, -0.25) {\(x_0\)};
            \draw (0, -1) rectangle (2.5, -1.5);
            \node at (1.25, -1.25) {\(y_0\)};
            \draw (3, 0) rectangle (6, -0.5);
            \node at (4.5, -0.25) {\(x_1\)};
            \draw (3, -1) rectangle (3.5, -1.5);
            \node at (3.25, -1.25) {\(y_1\)};
        \end{tikzpicture}
        
        \begin{tikzpicture}[
            semithick
        ]
            \def\xnul#1{
                \draw (#1 + 0, 0) rectangle (#1 + 1, -0.5);
                \node at (#1 + 0.5, -0.25) {\(x_0\)};
            }
            \def\xone#1{
                \draw (#1 + 0, 0) rectangle (#1 + 3, -0.5);
                \node at (#1 + 1.5, -0.25) {\(x_1\)};
            }
            \def\ynul#1{
                \draw (#1 + 0, 0) rectangle (#1 + 1, -0.5);
                \node at (#1 + 0.5, -0.25) {\(y_0\)};
            }
            \def\ynul#1{
                \draw (#1 + 0, -1) rectangle (#1 + 2.5, -1.5);
                \node at (#1 + 1.25, -1.25) {\(y_0\)};
            }
            \def\yone#1{
                \draw (#1 + 0, -1) rectangle (#1 + 0.5, -1.5);
                \node at (#1 + 0.25, -1.25) {\(y_1\)};
            }
            \xnul{0.0}
            \xnul{1.0}
            \xnul{2.0}
            \xnul{3.0}
            \xnul{4.0}
            \xone{5.0}
            \xone{8.0}
            \xone{11.0}
            \ynul{0.0}
            \ynul{2.5}
            \ynul{5.0}
            \ynul{7.5}
            \ynul{10.0}
            \yone{12.5}
            \yone{13}
            \yone{13.5}
        \end{tikzpicture}
    \end{minipage}

    \begin{minipage}{\linewidth}
        \centering
        \vspace{\distance}
        \begin{heading}
            On the Halting Problem
        \end{heading}

        \newcommand{\qj}{q_{ja}}
        \newcommand{\qn}{q_{nein}}
        \begin{tikzpicture}[
            semithick
        ]
            \node at (0.75, -0.75) {\(M'\)};
            \draw[->] (1.25, -1.75) -- (1.25, -2.25);
            \draw (0, 0) rectangle (10, -5);
            \draw (1, -2.25) rectangle (1.5, -2.75);
            \node at (1.25, -2.5) {\(\sigma_1\)};
            \draw (1.5, -2.25) rectangle (2, -2.75);
            \node at (1.75, -2.5) {\(\sigma_2\)};
            \draw (2, -2.25) rectangle (2.5, -2.75);
            \node at (2.25, -2.5) {\(...\)};
            \draw (2.5, -2.25) rectangle (3, -2.75);
            \node at (2.75, -2.5) {\(\sigma_n\)};
            \draw[->] (3, -2.5) -- (4.5, -2.5);
            \draw (4.5, -1) rectangle (9, -4);
            \node at (5.25, -1.75) {\(M\)};
            \draw[->] (9, -2.5) -- (12, -2.5);
            \node at (12.5, -2.5) {\(\qj\)};
            \path (6, -1) edge[->, out=90, in=90, above] node {Hält nicht} (7.5, -1);
        \end{tikzpicture}
    \end{minipage}
    
    \begin{minipage}{\linewidth}
        \centering
        \vspace{\distance}
        \begin{heading}
            Triangulation of a Triangle
        \end{heading}

        \begin{tikzpicture}[
            semithick
        ]
            \draw (0, 0) -- (-1.5, -3);
            \draw (0, 0) -- (2.5, -2);
            \draw (-1.5, -3) -- (2.5, -2);
            \draw (0, 0) -- (0, -2);
            \draw (0, -2) -- (-1.5, -3);
            \draw (0, -2) -- (2.5, -2);
            \node[above] at (0,0) {Q};
            \node[below left] at (-1.5,-3) {P};
            \node[right] at (2.5,-2) {R};
            \node[above left] at (0,-2) {T};
            \draw (-0.5, -2) -- (-1.5, -1.5);
            \draw (1, -1.5) -- (2, -0.5);
            \draw (0, -2.3) -- (0.5, -3.0);
            \node[left] at (-1.5,-1.5) {\(\triangle(T, R, Q)\)};
            \node[below right] at (0.5,-3.0) {\(\triangle(P, T, R)\)};
            \node[above right] at (2,-0.5) {\(\triangle(Q, R, T)\)};
            \node at (4,-1.5) {\(\triangle(P, Q, R)\)};
        \end{tikzpicture}
    \end{minipage}
    
    \begin{minipage}{\linewidth}
        \centering
        \vspace{\distance}
        \begin{heading}
            Matrices
        \end{heading}

        \(
            C = \left(\begin{array}{c|c}
                \begin{array}{ccc|ccc}
                    a_{11} & \cdots & ^* & ^* & \cdots & ^*\\
                    \vdots & \ddots & \vdots & \vdots & & \vdots\\
                    0 & \cdots & a_{rr} & ^* & \cdots & ^*\\
                    \hline
                    0 & \cdots & 0 & 0 & \cdots & 0\\
                    \vdots & & \vdots & \vdots & & \vdots\\
                    0 & \cdots & 0 & 0 & \cdots & 0
                \end{array} & \text{\huge 0}\\
                \hline
                \text{\huge 0} & \begin{array}{ccc|ccc}
                    b_{11} & \cdots & ^* & ^* & \cdots & ^*\\
                    \vdots & \ddots & \vdots & \vdots & & \vdots\\
                    0 & \cdots & b_{ss} & ^* & \cdots & ^*\\
                    \hline
                    0 & \cdots & 0 & 0 & \cdots & 0\\
                    \vdots & & \vdots & \vdots & & \vdots\\
                    0 & \cdots & 0 & 0 & \cdots & 0
                \end{array}
            \end{array}\right)
        \)

        \vspace{0.5cm}

        \(
            C' = \left(\begin{array}{cccccc|ccc}
                a_{11} & \cdots & \cdots & \cdots & \cdots & ^* & ^* & \cdots & ^*\\
                \vdots & \ddots & & & & \vdots & \vdots & & \vdots\\
                \vdots & & a_{rr} & & & \vdots & \vdots & & \vdots\\
                \vdots & & & b_{11} & & \vdots & \vdots & & \vdots\\
                \vdots & & & & \ddots & \vdots & \vdots & & \vdots\\
                0 & \cdots & \cdots & \cdots & \cdots & b_{ss} & ^* & \cdots & ^*\\
                \hline
                0 & \cdots & \cdots & \cdots & \cdots & 0 & 0 & \cdots & 0\\
                \vdots & & & & & \vdots & \vdots & & \vdots\\
                0 & \cdots & \cdots & \cdots & \cdots & 0 & 0 & \cdots & 0
            \end{array}\right)
        \)
    \end{minipage}
    
    \begin{minipage}{\linewidth}
        \centering
        \vspace{\distance}
        \begin{heading}
            Manchester Encoding
        \end{heading}

        \begin{tikzpicture}[
            scale=0.8,
            semithick
        ]
            \def\nul#1{
                \draw[dotted] (#1 + 0, -0.5) -- (#1 + 0, 1.5);
                \draw (#1 + 0, 0)      -- (#1 + 0.25, 0);
                \draw (#1 + 0.25, 0)   -- (#1 + 0.25, 0.5);
                \draw (#1 + 0.25, 0.5) -- (#1 + 0.5, 0.5);
                \draw (#1 + 0.5, 0.5)  -- (#1 + 0.5, 0);
                \node at (#1 + 0.25, 1) {\texttt{0}};
            }
            \def\nulconn#1{
                \draw[dotted] (#1 + 0, -0.5) -- (#1 + 0, 1.5);
                \draw (#1 + 0, 0)      -- (#1 + 0.25, 0);
                \draw (#1 + 0.25, 0)   -- (#1 + 0.25, 0.5);
                \draw (#1 + 0.25, 0.5) -- (#1 + 0.5, 0.5);
                \node at (#1 + 0.25, 1) {\texttt{0}};
            }
            \def\one#1{
                \draw[dotted] (#1 + 0, -0.5) -- (#1 + 0, 1.5);
                \draw (#1 + 0, 0)      -- (#1 + 0, 0.5);
                \draw (#1 + 0, 0.5)    -- (#1 + 0.25, 0.5);
                \draw (#1 + 0.25, 0.5) -- (#1 + 0.25, 0);
                \draw (#1 + 0.25, 0)   -- (#1 + 0.5, 0);
                \node at (#1 + 0.25, 1) {\texttt{1}};
            }
            \def\oneconn#1{
                \draw[dotted] (#1 + 0, -0.5) -- (#1 + 0, 1.5);
                \draw (#1 + 0, 0.5)    -- (#1 + 0.25, 0.5);
                \draw (#1 + 0.25, 0.5) -- (#1 + 0.25, 0);
                \draw (#1 + 0.25, 0)   -- (#1 + 0.5, 0);
                \node at (#1 + 0.25, 1) {\texttt{1}};
            }
            \one{0.5}
            \one{1.0}
            \one{1.5}
            \one{2.0}
            \one{2.5}
            \nulconn{3.0}
            \oneconn{3.5}
            \nul{4.0}
            \nul{4.5}
            \nulconn{5.0}
            \oneconn{5.5}
            \one{6.0}
            \one{6.5}
            \one{7.0}
            \one{7.5}
            \nulconn{8.0}
            \oneconn{8.5}
            \nulconn{9.0}
            \oneconn{9.5}
            \one{10.0}
            \nul{10.5}
            \nul{11.0}
            \nul{11.5}
            \nulconn{12.0}
            \oneconn{12.5}
            \nulconn{13.0}
            \oneconn{13.5}
            \draw[dotted] (14, -0.5) -- (14, 1.5);
        \end{tikzpicture}
    \end{minipage}
    
    \begin{minipage}{\linewidth}
        \centering
        \vspace{\distance}
        \begin{heading}
            Network Package Routing
        \end{heading}

        \begin{tikzpicture}[
            semithick
        ]
            \draw (0, 0) rectangle node {\texttt{5.1.1.1}} (2, -1);
            \draw (5, 0) rectangle node {\texttt{4.1.1.1}} (7, -1);
            \draw (0, -2) rectangle node {\texttt{3.1.1.1}} (2, -3);
            \draw (5, -2) rectangle node {\texttt{2.1.1.1}} (7, -3);
            \draw (2.5, -4) rectangle node {\texttt{1.1.1.1}} (4.5, -5);
            \draw[blue!100, ->] (5, -0.5) -- (2, -0.5);
            \draw (1, -1) -- (1, -2);
            \draw[blue!100, ->] (2, -2) -- (5, -1);
            \draw (5, -2.5) -- (2, -2.5);
            \draw (4.5, -4) -- (5, -3);
            \draw[blue!100, ->] (2.5, -4) -- (2, -3);
        \end{tikzpicture}

        \begin{tikzpicture}[
            semithick
        ]
            \draw (0, 0) rectangle node {\texttt{5.1.1.1}} (2, -1);
            \draw (5, 0) rectangle node {\texttt{4.1.1.1}} (7, -1);
            \draw (0, -2) rectangle node {\texttt{3.1.1.1}} (2, -3);
            \draw (5, -2) rectangle node {\texttt{2.1.1.1}} (7, -3);
            \draw (2.5, -4) rectangle node {\texttt{1.1.1.1}} (4.5, -5);
            \draw (5, -0.5) -- (2, -0.5);
            \draw (1, -1) -- (1, -2);
            \draw (2, -2) -- (5, -1);
            \draw[blue!100, ->] (2, -2.5) -- (5, -2.5);
            \draw[blue!100, ->] (5, -3) -- (4.5, -4);
            \draw[blue!100, ->] (2.5, -4) -- (2, -3);
        \end{tikzpicture}
    \end{minipage}

    {
        \vspace{\distance}
        \begin{heading}
            Swapping Algorithm Description
        \end{heading}
        \begin{algorithm}
            \caption{Swapping values}
            \begin{algorithmic}[1]
                \Procedure{swap}{$a, b$}
                    \State \(t \leftarrow a\)
                    \State \(a \leftarrow b\)
                    \State \(b \leftarrow t\)
                \EndProcedure
            \end{algorithmic}
        \end{algorithm}
    }

    \begin{minipage}{\linewidth}
        \centering
        \vspace{\distance}
        \begin{heading}
            Binary Trees
        \end{heading}

        \begin{tikzpicture}[
            scale=2.5,
            every node/.style={circle, draw, inner sep=1mm}
        ]
            \coordinate (0) at (-0.25, -1);
            \draw (0, 0) -- (0.5,-1) -- (-0.5,-1) -- (0, 0);
            \node (1) at (-0.25, -1.5) {};
            \draw (0) -- (1);
            \node (2) at (-0.5, -1.75) {};
            \node (3) at (0, -1.75) {};
            \draw (1) -- (2);
            \draw (1) -- (3);
            \node[draw=none] at (1, -1) {\huge\(\leadsto\)};
            \coordinate (4) at (1.75, -1);
            \draw (2, 0) -- (2.5,-1) -- (1.5,-1) -- (2, 0);
            \node (5) at (1.75, -1.5) {};
            \draw (4) -- (5);
            \node (6) at (1.5, -1.75) {};
            \node (7) at (2, -1.75) {};
            \draw (5) -- (6);
            \draw (5) -- (7);
            \node (8) at (1.75, -2) {};
            \node (9) at (2.25, -2) {};
            \draw (7) -- (8);
            \draw (7) -- (9);
        \end{tikzpicture}
    \end{minipage}

    \begin{minipage}{\linewidth}
        \centering
        \vspace{\distance}
        \begin{heading}
            Skip List
        \end{heading}

        \begin{tikzpicture}[
                semithick
            ]
            \tikzset{every node/.style={inner sep=0cm,minimum size=0.5cm}}

            \def\h{1}
            \draw (0, \h) circle (0.25) node[left=0.25cm, inner sep=0.25cm] {\(h=\) \h};
            \draw (0.75, -0.25+\h) rectangle node (01) {1} (1.25,  \h+0.25);
            \draw (1.75, -0.25+\h) rectangle node (02) {3} (2.25,  \h+0.25);
            \draw (2.75, -0.25+\h) rectangle node (03) {5} (3.25,  \h+0.25);
            \draw (3.75, -0.25+\h) rectangle node (04) {6} (4.25,  \h+0.25);
            \draw (4.75, -0.25+\h) rectangle node (05) {8} (5.25,  \h+0.25);
            \draw (5.75, -0.25+\h) rectangle node (06) {9} (6.25,  \h+0.25);
            \draw (6.75, -0.25+\h) rectangle node (07) {10} (7.25, \h+0.25);
            \draw[-|] (7.25, \h) -- (7.5, \h);
            \path[->]
            (0.25, \h) edge (01)
            (01) edge (02)
            (02) edge (03)
            (03) edge (04)
            (04) edge (05)
            (05) edge (06)
            (06) edge (07);

            \def\h{2}
            \draw (0, \h) circle (0.25) node[left=0.25cm, inner sep=0.25cm] {\(h=\) \h};
            \draw (1.75, -0.25+\h) rectangle node (12) {3} (2.25,  \h+0.25);
            \draw (2.75, -0.25+\h) rectangle node (13) {5} (3.25,  \h+0.25);
            \draw (3.75, -0.25+\h) rectangle node (14) {6} (4.25,  \h+0.25);
            \draw (4.75, -0.25+\h) rectangle node (15) {8} (5.25,  \h+0.25);
            \draw[-|] (5.25, \h) -- (5.5, \h);
            \path[->]
            (0, -0.25+\h) edge (0, -0.75+\h)
            (0.25, \h) edge (12)
            (12) edge (13)
            (13) edge (14)
            (14) edge (15)
            (12) edge (02)
            (13) edge (03)
            (14) edge (04)
            (15) edge (05);

            \def\h{3}
            \draw (0, \h) circle (0.25) node[left=0.25cm, inner sep=0.25cm] {\(h=\) \h};
            \draw (2.75, -0.25+\h) rectangle node (23) {5} (3.25,  \h+0.25);
            \draw (3.75, -0.25+\h) rectangle node (24) {6} (4.25,  \h+0.25);
            \draw (4.75, -0.25+\h) rectangle node (25) {8} (5.25,  \h+0.25);
            \draw[-|] (5.25, \h) -- (5.5, \h);
            \path[->]
            (0, -0.25+\h) edge (0, -0.75+\h)
            (0.25, \h) edge (23)
            (23) edge (24)
            (24) edge (25)
            (23) edge (13)
            (24) edge (14)
            (25) edge (15);

            \def\h{4}
            \draw (0, \h) circle (0.25) node[left=0.25cm, inner sep=0.25cm] {\(h=\) \h};
            \draw (2.75, -0.25+\h) rectangle node (33) {5} (3.25,  \h+0.25);
            \draw (4.75, -0.25+\h) rectangle node (35) {8} (5.25,  \h+0.25);
            \draw[-|] (5.25, \h) -- (5.5, \h);
            \path[->]
            (0, -0.25+\h) edge (0, -0.75+\h)
            (0.25, \h) edge (33)
            (33) edge (35)
            (33) edge (23)
            (35) edge (25);

            \def\h{5}
            \draw (0, \h) circle (0.25) node[left=0.25cm, inner sep=0.25cm] {\(h=\) \h};
            \draw (4.75, -0.25+\h) rectangle node (45) {8} (5.25,  \h+0.25);
            \draw[-|] (5.25, \h) -- (5.5, \h);
            \path[->]
            (0, -0.25+\h) edge (0, -0.75+\h)
            (0.25, \h) edge (45)
            (45) edge (35);

            \def\h{6}
            \draw (0, \h) circle (0.25) node[left=0.25cm, inner sep=0.25cm] {\(h=\) \h};
            \draw (4.75, -0.25+\h) rectangle node (55) {8} (5.25,  \h+0.25);
            \draw[-|] (5.25, \h) -- (5.5, \h);
            \path[->]
            (0, -0.25+\h) edge (0, -0.75+\h)
            (0.25, \h) edge (55)
            (55) edge (45);
        \end{tikzpicture}
    \end{minipage}

    % Inspiration from the paper "Continuous K Nearest Neighbor Queries in SpatialNetwork Databases" by Mohammad R. Kolahdouzan and Cyrus Shahabi
    \begin{minipage}{\linewidth}
        \centering
        \vspace{\distance}
        \begin{heading}
            Countinuous First Neighbors
        \end{heading}

        \begin{tikzpicture}[
            semithick,
            scale=0.75,
            every node/.style={circle, fill, inner sep=0pt, minimum size=2pt}
        ]
            \footnotesize
            \draw (-5, -3) rectangle (5, 3);
            \path[draw] (-4, 3) -- (-1, 0) -- (-3, -3);
            \path[draw] (-1, 0) -- (-0.5, 0) -- (1, 1.5) -- (2, 0) -- (1, -1.5) -- (-0.5, 0);
            \draw (1, 1.5) -- (1.5, 3);
            \draw (1, -1.5) -- (1.5, -3);
            \draw (2, 0) -- (3, 0);
            \draw (3, 0) -- (4, 3);
            \draw (3, 0) -- (5, -1.5);
            \node[label=below left:{\(s\)}] (s) at (-4, 1.5) {};
            \node[label=above right:{\(p_1\)}] at (-1, 2) {};
            \node[label=below left:{\(p_2\)}] at (-3, -1) {};
            \node[label=below:{\(p_3\)}] at (-0.5, -1.5) {};
            \node[label=above right:{\(p_4\)}] at (0.75, 0) {};
            \node[label=below right:{\(p_5\)}] at (2.5, -1.25) {};
            \node[label=above right:{\(p_6\)}] at (2.25, 1.5) {};
            \node[label=above right:{\(p_7\)}] at (4.25, 1) {};
            \node[label=above right:{\(t\)}] (t) at (2.15, -0.5) {};
            \draw[dashed] plot[smooth, tension=1] coordinates { (s) (-1.25, 1.75) (-2.75, -0.75) (-0.5, -1.25) (0.75, -0.25) (t) };
        \end{tikzpicture}
    \end{minipage}

    \begin{minipage}{\linewidth}
        \centering
        \vspace{\distance}
        \begin{heading}
            Rotation in der Ebene
        \end{heading}

        \begin{center}
            \(
                \begin{pmatrix}
                    x\\
                    y
                \end{pmatrix} = \begin{pmatrix}
                    x\\
                    0
                \end{pmatrix} + \begin{pmatrix}
                    0\\
                    y
                \end{pmatrix} \mapsto \begin{pmatrix}
                    \cos(\psi) \cdot x - \sin(\psi) \cdot y\\
                    \sin(\psi) \cdot x + \cos(\psi) \cdot y
                \end{pmatrix}
            \)
        \end{center}

        \begin{tikzpicture}[
            >=stealth,
            semithick
        ]
            % Axis
            \draw[->] (0, 0) -- (0, 5);
            \draw[->] (-5, 0) -- (5, 0);

            \draw[->] (0, 0) -- (4, 0);
            \draw[->] (0, 0) -- (0, 4);
            \draw (0, 0) -- (2.83, 0) node[below, pos=0.5] {\(\cos{\psi}\)};
            \draw (0, 0) -- (0, 2.83) node[right, pos=0.5, rotate=270, anchor=south] {\(\cos{\psi}\)};
            \draw (2.83, 2.83) -- (2.83, 0) node[right, pos=0.5, rotate=270, anchor=south] {\(\sin{\psi}\)};
            \draw (-2.83, 2.83) -- (0, 2.83) node[above, pos=0.5] {\(-\sin{\psi}\)};
            % Calculations result from rotating the vectors for
            % the angle psi = 45 degrees and then adjusting the magnitude
            % for the text of the angle variable for instance
            \draw[->] (0, 0) -- (2.83, 2.83) node[above left, pos=0.5] {\(1\)};
            \draw[->] (0, 0) -- (-2.83, 2.83) node[below left, pos=0.5] {\(1\)};
            \draw (1.25, 0) arc (0:45:1.25);
            \node at (0.70, 0.29) {\(\psi\)};
            \draw (0, 1.25) arc (90:135:1.25);
            \draw[->] (4, 0) arc (0:45:4);
            \draw[->] (0, 4) arc (90:135:4);
            \node at (-0.29, 0.70) {\(\psi\)};
            \node[below] at (4, 0) {\((1, 0)\)};
            \node[right] at (0, 4) {\((0, 1)\)};
        \end{tikzpicture}
    \end{minipage}
    \newpage
    \begin{minipage}{\linewidth}
        \centering
        \vspace{\distance}
        \begin{heading}
            Three-dimensional Borel Sets
        \end{heading}
        \vspace{0.25cm}
    \end{minipage}
    \begin{minipage}{0.5\linewidth}
        \centering
        \begin{tikzpicture}[
            >=stealth,
            scale=0.75,
            semithick
        ]
            % Line must be of length 3 => sqrt(2) x = 3 => x = 2.12
            \draw[->] (-1.06, -1.06) -- (-2.12, -2.12);
            \draw[->] (2.5, 0) -- (6, 0);
            \draw[->] (0, 2.5) -- (0, 6);
            \fill[gray!10] (0, 0) -- (2.5, 0) arc (0:90:2.5) -- cycle;
            % Bezier Curves https://www.tug.org/pracjourn/2007-1/mertz/mertz.pdf p. 12
            \fill[gray!10] (0, 0) -- (0, 2.5) .. controls (-1.06, 1.44) and (-1.06, -0.06) .. (-1.06, -1.06) -- cycle;
            \fill[gray!10] (0, 0) -- (2.5, 0) .. controls (1.44, -1.06) and (-0.06, -1.06) .. (-1.06, -1.06) -- cycle;
            \draw (2.5, 0) arc (0:90:2.5);
            \draw (0, 2.5) .. controls (-1.06, 1.44) and (-1.06, -0.06) .. (-1.06, -1.06);
            \draw (2.5, 0) .. controls (1.44, -1.06) and (-0.06, -1.06) .. (-1.06, -1.06);
            \draw[dashed] (0, 0) -- (-1.06, -1.06);
            \draw[dashed] (0, 0) -- (2.5, 0);
            \draw[dashed] (0, 0) -- (0, 2.5);
            \draw (-1.77, -1.77) -- (3.23, -1.77) -- (3.23, 3.23) -- (-1.77, 3.23) -- cycle;
            \draw (3.23, -1.77) -- (5, 0) -- (5, 5) -- (3.23, 3.23) -- cycle;
            \draw (-1.77, 3.23) -- (3.23, 3.23) -- (5, 5) -- (0, 5) -- cycle;
            \node at (0.72, 0.72) {\(A_1\)};
            \node at (3, 4) {\([0,2]^3\)};
        \end{tikzpicture}
    \end{minipage}
    \begin{minipage}{0.5\linewidth}
        \centering
        \begin{tikzpicture}[
            >=stealth,
            scale=0.75,
            semithick
        ]
            \draw[->] (-1.06, -1.06) -- (-2.12, -2.12);
            \draw[->] (2.5, 0) -- (6, 0);
            \draw[->] (0, 2.5) -- (0, 6);
            \fill[gray!10] (0, 0) -- (-1.06, -1.06) -- (1.44, -1.06) -- (2.5, 0) -- cycle;
            \fill[gray!10] (0, 0) -- (-1.06, -1.06) -- (-1.06, 1.44) -- (0, 2.5) -- cycle;
            \fill[gray!10] (0, 0) -- (2.5, 0) -- (2.5, 2.5) -- (0, 2.5) -- cycle;
            \draw[dashed] (0, 0) -- (-1.06, -1.06);
            \draw[dashed] (0, 0) -- (2.5, 0);
            \draw[dashed] (0, 0) -- (0, 2.5);
            \draw (-1.06, -1.06) -- (1.44, -1.06) -- (1.44, 1.44) -- (-1.06, 1.44) -- cycle;
            \draw (-1.06, 1.44) -- (0, 2.5) -- (2.5, 2.5) -- (1.44, 1.44) -- cycle;
            \draw (1.44, -1.06) -- (2.5, 0) -- (2.5, 2.5) -- (1.44, 1.44) -- cycle;
            \draw (-1.77, -1.77) -- (3.23, -1.77) -- (3.23, 3.23) -- (-1.77, 3.23) -- cycle;
            \draw (3.23, -1.77) -- (5, 0) -- (5, 5) -- (3.23, 3.23) -- cycle;
            \draw (-1.77, 3.23) -- (3.23, 3.23) -- (5, 5) -- (0, 5) -- cycle;
            \node at (0.72, 0.72) {\(\overline{A_2}\)};
            \node at (3, 4) {\([0,2]^3\)};
        \end{tikzpicture}
    \end{minipage}
    \begin{minipage}{0.5\linewidth}
        \centering
        \begin{tikzpicture}[
            >=stealth,
            scale=0.75,
            semithick
        ]
            \draw[->] (-1.06, -1.06) -- (-2.12, -2.12);
            \draw[->] (2.5, 0) -- (6, 0);
            \draw[->] (0, 2.5) -- (0, 6);
            \fill[gray!10] (-1.06, -1.06) -- (2.5, 0) -- (0, 2.5) -- cycle;
            \draw[dashed] (0, 0) -- (-1.06, -1.06);
            \draw[dashed] (0, 0) -- (2.5, 0);
            \draw[dashed] (0, 0) -- (0, 2.5);
            \draw (-1.06, -1.06) -- (2.5, 0) -- (0, 2.5) -- cycle;
            \draw (-1.77, -1.77) -- (3.23, -1.77) -- (3.23, 3.23) -- (-1.77, 3.23) -- cycle;
            \draw (3.23, -1.77) -- (5, 0) -- (5, 5) -- (3.23, 3.23) -- cycle;
            \draw (-1.77, 3.23) -- (3.23, 3.23) -- (5, 5) -- (0, 5) -- cycle;
            \node at (0.72, 0.72) {\(A_3\)};
            \node at (3, 4) {\([0,2]^3\)};
        \end{tikzpicture}
    \end{minipage}
    \begin{minipage}{0.5\linewidth}
        \centering
        \begin{tikzpicture}[
            >=stealth,
            scale=0.75,
            semithick
        ]
            \draw[->] (-1.77, -1.77) -- (-2.12, -2.12);
            \draw[->] (5, 0) -- (6, 0);
            \draw[->] (0, 5) -- (0, 6);
            \fill[gray!10] (-1.77, -1.77) -- (5, 0) -- (0, 5) -- cycle;
            \draw[dashed] (0, 0) -- (-1.77, -1.77);
            \draw[dashed] (0, 0) -- (5, 0);
            \draw[dashed] (0, 0) -- (0, 5);
            \draw (-1.77, -1.77) -- (5, 0) -- (0, 5) -- cycle;
            \draw (-1.77, -1.77) -- (3.23, -1.77) -- (3.23, 3.23) -- (-1.77, 3.23) -- cycle;
            \draw (3.23, -1.77) -- (5, 0) -- (5, 5) -- (3.23, 3.23) -- cycle;
            \draw (-1.77, 3.23) -- (3.23, 3.23) -- (5, 5) -- (0, 5) -- cycle;
            \node at (1.25, 1.25) {\(A_4\)};
            \node at (3, 4) {\([0,2]^3\)};
        \end{tikzpicture}
    \end{minipage}
\end{document}
